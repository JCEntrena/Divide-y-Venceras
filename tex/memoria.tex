% %%%
% Modificación de una plantilla de Latex para adaptarla al castellano.
%%%

%%%%%%%%%%%%%%%%%%%%%%%%%%%%%%%%%%%%%%%%%
% Thin Sectioned Essay
% LaTeX Template
% Version 1.0 (3/8/13)
%
% This template has been downloaded from:
% http://www.LaTeXTemplates.com
%
% Original Author:
% Nicolas Diaz (nsdiaz@uc.cl) with extensive modifications by:
% Vel (vel@latextemplates.com)
%
% License:
% CC BY-NC-SA 3.0 (http://creativecommons.org/licenses/by-nc-sa/3.0/)
%
%%%%%%%%%%%%%%%%%%%%%%%%%%%%%%%%%%%%%%%%%

%----------------------------------------------------------------------------------------
%	PACKAGES AND OTHER DOCUMENT CONFIGURATIONS
%----------------------------------------------------------------------------------------

\documentclass[a4paper, 11pt]{article} % Font size (can be 10pt, 11pt or 12pt) and paper size (remove a4paper for US letter paper)

\usepackage[protrusion=true,expansion=true]{microtype} % Better typography
\usepackage{graphicx} % Required for including pictures
\usepackage[usenames,dvipsnames]{color} % Coloring code
\usepackage{wrapfig} % Allows in-line images
\usepackage[utf8]{inputenc}

% sudo apt-get install texlive-lang-spanish
\usepackage[spanish]{babel} % English language/hyphenation
\selectlanguage{spanish}
% Hay que pelearse con babel-spanish para el alineamiento del punto decimal
\decimalpoint
\usepackage{dcolumn}
\newcolumntype{d}[1]{D{.}{\esperiod}{#1}}
\makeatletter
\addto\shorthandsspanish{\let\esperiod\es@period@code}
\makeatother

\usepackage{longtable}
\usepackage{tabu}
\usepackage{supertabular}

\usepackage{multicol}
\newsavebox\ltmcbox

% Para algoritmos
\usepackage{algorithm}
\usepackage{algorithmic}
\floatname{algorithm}{Algoritmo}
\renewcommand{\listalgorithmname}{Lista de algoritmos}
\renewcommand{\algorithmicrequire}{\textbf{Entrada:}}
\renewcommand{\algorithmicensure}{\textbf{Salida:}}
\renewcommand{\algorithmicend}{\textbf{fin}}
\renewcommand{\algorithmicif}{\textbf{si}}
\renewcommand{\algorithmicthen}{\textbf{entonces}}
\renewcommand{\algorithmicelse}{\textbf{en otro caso}}
\renewcommand{\algorithmicelsif}{\algorithmicelse,\ \algorithmicif}
\renewcommand{\algorithmicendif}{\algorithmicend\ \algorithmicif}
\renewcommand{\algorithmicfor}{\textbf{para }}
\renewcommand{\algorithmicforall}{\textbf{para cada}}
\renewcommand{\algorithmicdo}{\textbf{}}
\renewcommand{\algorithmicendfor}{\algorithmicend\ \algorithmicfor}
\renewcommand{\algorithmicwhile}{\textbf{mientras}}
\renewcommand{\algorithmicendwhile}{\algorithmicend\ \algorithmicwhile}
\renewcommand{\algorithmicloop}{\textbf{repetir}}
\renewcommand{\algorithmicendloop}{\algorithmicend\ \algorithmicloop}
\renewcommand{\algorithmicrepeat}{\textbf{repetir}}
\renewcommand{\algorithmicuntil}{\textbf{hasta que}}
\renewcommand{\algorithmicprint}{\textbf{imprimir}} 
\renewcommand{\algorithmicreturn}{\textbf{devolver}} 
\renewcommand{\algorithmictrue}{\textbf{true }} 
\renewcommand{\algorithmicfalse}{\textbf{false }} 
\renewcommand{\algorithmicand}{\textbf{y}}
\renewcommand{\algorithmicor}{\textbf{o}}


\usepackage[section]{placeins} % Para gráficas en su sección.
\usepackage{mathpazo} % Use the Palatino font
\usepackage[T1]{fontenc} % Required for accented characters
\newenvironment{allintypewriter}{\ttfamily}{\par}
\setlength{\parindent}{0pt}
\parskip=8pt
\linespread{1.05} % Change line spacing here, Palatino benefits from a slight increase by default

\makeatletter
\renewcommand\@biblabel[1]{\textbf{#1.}} % Change the square brackets for each bibliography item from '[1]' to '1.'
\renewcommand{\@listI}{\itemsep=0pt} % Reduce the space between items in the itemize and enumerate environments and the bibliography
\newcommand{\imagen}[2]{\begin{center} \includegraphics[width=90mm]{#1} \\#2 \end{center}}

\renewcommand{\maketitle}{ % Customize the title - do not edit title and author name here, see the TITLE block below
\begin{flushright} % Right align
{\LARGE\@title} % Increase the font size of the title

\vspace{50pt} % Some vertical space between the title and author name

{\large\@author} % Author name
\\\@date % Date

\vspace{40pt} % Some vertical space between the author block and abstract
\end{flushright}
}

%----------------------------------------------------------------------------------------
%	TITLE
%----------------------------------------------------------------------------------------

\title{\textbf{Práctica 2}\\ % Title
Divide y vencerás} % Subtitle

\author{\textsc{Óscar Bermúdez,\\Francisco David Charte,\\Ignacio Cordón,\\José Carlos Entrena,\\Mario Román} % Author
\\{\textit{Universidad de Granada}}} % Institution

\date{\today} % Date

%----------------------------------------------------------------------------------------

\begin{document}

\maketitle % Print the title section

\renewcommand{\abstractname}{Resumen} % Uncomment to change the name of the abstract to something else
\begin{abstract}

\end{abstract}
{\parskip=2pt
\tableofcontents
}
\pagebreak

\section {Suma de $n$ números}
\subsection{Algoritmo}
\begin{algorithm}[H]
	\begin{algorithmic}[1]
		\REQUIRE \ \\
        	\texttt{v}, lista de $n$ números\\
     	\IF{$n$==1}
	  \RETURN {\texttt{v[0]}}
	\ELSE
	  \RETURN {\texttt{Suma(\{v[1]\ldots v[n/2]\}) + Suma (\{v[n/2+1]\ldots v[n]\})}}
    	\ENDIF \\\
	\end{algorithmic}
    \caption{Suma de $n$ números}
    \label{suma}
\end{algorithm}

\subsection{Análisis de eficiencia}
El algoritmo responde a una recurrencia de la forma:
\begin{equation}
 T(n)=\left\lbrace
	    \begin{array}{lr}
            1 & n=1\\
            2\cdot T\left(\frac{n}{2}\right) & n>1\\
            \end{array}
	    \right.
\end{equation}

Resolviendo la recurrencia, tenemos:
\begin{equation}
 T(n)=n\quad \forall n\in \mathbb{N}
\end{equation}

Es claro que al tener que recorrer todas las componentes (es necesario leer las posiciones
del vector, para poder sumarlas), el mejor algoritmo que se puede ejecutar en un único core 
será de complejidad $\mathcal{O}(n)$. En el caso de un único core, por tanto, no supone ninguna
mejora el realizar el algoritmo Divide y Vencerás; más aún, este algoritmo será más lento que el
iterativo, por la necesidad de crear marcos de pila nuevos para cada llamada a la función.\\
% (Comentario Mario)
% La idea de aprovechar con un SIMD se aplica cuando el algoritmo queda escrito de abajo a arriba.
% Pero se puede aprovechar un SIMD.
Para un procesador vectorial (SIMD), de nuevo encontramos que el algoritmo iterativo convencional
(recorrer todo el vector e ir acumulando en una variable suma) sería más eficiente que el Divide y Vencerás.
Únicamente encontraríamos mejora si el computador fuese de tipo MIMD, ya que en dicho caso podríamos crear
varias hebras y que cada una de ellas se ocupara de atender una llamada a la recurrencia en un procesador distinto.

\subsection{Implementación}
A continuación mostramos una implementación en el lenguaje Ruby:

\small
\texttt{% Generator: GNU source-highlight, by Lorenzo Bettini, http://www.gnu.org/software/src-highlite
\noindent
\mbox{}\textbf{\textcolor{Blue}{class}}\ Array \\
\mbox{}\ \ \textbf{\textcolor{Blue}{def}}\ sum \\
\mbox{}\ \ \ \ \textbf{\textcolor{Blue}{return}}\ first\ \textbf{\textcolor{Blue}{if}}\ size\ \textcolor{BrickRed}{\textless{}}\ \textcolor{Purple}{2} \\
\mbox{} \\
\mbox{}\ \ \ \ \textbf{\textcolor{Blue}{self}}\textcolor{BrickRed}{[}\textcolor{Purple}{0}\ \textcolor{BrickRed}{..}\ size\texttt{\textcolor{Orange}{/2-1].sum\ +\ self[size/}}\textcolor{Purple}{2}\ \textcolor{BrickRed}{..}\ size\ \textcolor{BrickRed}{-}\ \textcolor{Purple}{1}\textcolor{BrickRed}{].}sum \\
\mbox{}\ \ \textbf{\textcolor{Blue}{end}} \\
\mbox{}\textbf{\textcolor{Blue}{end}} \\
\mbox{}}
\normalsize

\section {Multiplicación de números grandes (adaptado a 
binario)}
\subsection{Algoritmo}
Llamamos longitud en bits de un número entero positivo $x=x_{n}x_{n-1}\ldots x_1)_b \quad \\ x_i \in \{0,1\}$
a $\max\{i\in \{1\ldots n\} : x_i = 1\}$
% (Comentario Mario)
% ¿Por qué usamos las rotaciones en lugar de desplazamientos?
Denotamos por \texttt{x $>>$ y} y \texttt{x $<<$ y} a las rotaciones a derecha e izquierda $y$ posiciones
de los bits de un número entero $x$.
\begin{algorithm}[H]
	\begin{algorithmic}[1]
	  \REQUIRE \ \\
	  \texttt{x,y, números enteros positivos}\\\
        	
	  % (Comentario Mario)
	  % ¿Y sustituir esto por signo = sgn(xy), o algo parecido?
	  \IF{[x<0 \AND y<0] \OR [x>0 \AND y>0]}
	    \STATE{\texttt{signo=}1}
	  \ELSE
	    \STATE{\texttt{signo=}-1}
	  \ENDIF\\\
	  \STATE{\texttt{(x,y):=(|x|,|y|)}}
	  \STATE{\texttt{n}:= menor de las longitudes en bits de \texttt{x,y}}
	  \\\
	  \IF{\texttt{longitud} $<$ 2}
	    \RETURN{$x\cdot y \cdot \texttt{signo}$}
	  \ELSE
	    \STATE{\texttt{(a,c):= $\left\lceil \frac{n}{2}\right\rceil$} bits superiores de \texttt{(x,y)}}
	    \STATE{\texttt{(b,d):= $\left\lfloor \frac{n}{2}\right\rfloor$} bits inferiores de \texttt{(x,y)}}
	    \STATE{Calculamos \texttt{a$\cdot$c}, \texttt{b$\cdot$d} y \texttt{(a-b)$\cdot$(d-c)} llamando recursivamente al algoritmo}
	  \ENDIF \\\
	  \RETURN\\\ $\bigg( ac << (\left\lceil n/2 \right\rceil \cdot 2) +
	    \bigg[(a-b)(d-c) +ac +bd\bigg] << \left\lfloor n/2\right\rfloor + 
	    bd \bigg) \cdot \texttt{signo}$
	  
	\end{algorithmic}
    \caption{Algoritmo de Karatsuba}
    \label{multiplicacion}
\end{algorithm}
\newpage
Por ejemplo, para los números \texttt{x = 10011011, y = 10111010}, el producto por el algoritmo:
\begin{itemize}
\item Son ambos positivos, luego \texttt{signo=1}
\item \texttt{(a,c)=(1001,1011)}
\item \texttt{(b,d)=(1011,1010)}
\item \texttt{ac=1100011, bd=1101110 y (a-b)(d-c)=10}
\item Devolver \texttt{ 110001100000000 + (1100011+1101110+10)0000 + 1101110 = 111000010011110}
\end{itemize} 

\subsection{Análisis de eficiencia}
El algoritmo convencional de multiplicación de dos enteros de $n$ cifras, tiene una eficiencia de $\mathcal{O}(n^2)$
El algoritmo sigue la siguiente recurrencia:
\begin{equation}\label{rec}
 T(n)=\left\lbrace
	    \begin{array}{lr}
            1 & n=1\\
            3\cdot T\left(\frac{n}{2}\right) + M\cdot n & n>1\\
            \end{array}
	    \right.
\end{equation}

donde $M\in\mathbb{R}$ es una constante, y el sumando $n$ del caso $n>1$ de la recurrencia viene dado por el cálculo
de la longitud de los números en bits, las sumas y restas que se efectúan en el algoritmo, y las rotaciones.

Para resolver la recurrencia, efectuamos el cambio de variable $n=2^k$:
\begin{equation}\label{prodcv}
 T(2^k)=3\cdot T(2^{k-1}) + M\cdot 2^k
\end{equation}
y resolvemos: 

\begin{equation}
x_k=3\cdot x_{k-1}+M\cdot 2^k
\end{equation}
Calculamos una solución particular de la recurrencia no homogénea:
$$A\cdot 2^k - 3\cdot A \cdot 2^{k-1}=-A\cdot 2^{k-1}=M2^k \leftrightarrow A=-2M$$
y por tanto: $x_k=-2M \cdot 2^k$ es solución particular.\\
Calculamos ahora la solución general a la recurrencia homogénea:
$$ x_k-3\cdot x_{k-1}=0$$
que tiene por polinomio característico a $x-3$. Por tanto, la solución general de la
recurrencia homogénea es:
$$ x_k= p\cdot 3^k\qquad p\in\mathbb{R}$$
\\\
Concluimos que la solución a la recurrencia \ref{prodcv} es: 
\begin{equation}
x_k = p \cdot 2^k + q \cdot 3^k \qquad p,q \in \mathbb{R}
\end{equation}
y deshaciendo el cambio en variable, llegamos a la solución de \ref{rec}:
\begin{eqnarray*}
T(n)=p \cdot 2^{log_2(n)} + q \cdot 3^{log_2(n)} = p \cdot 2^{\frac{log(n)}{log(2)}} +\\
q \cdot 3^{\frac{log(n)}{log(2)}}=p \cdot n + q \cdot n^{\frac{log(3)}{log(2)}} \quad \in \mathcal{O}(n^{1.585})
\end{eqnarray*}

Por tanto el algoritmo de Karatsuba es mejor asintóticamente hablando que el convencional para efectuar
el producto de dos números

\subsection{Implementación}
A continuación, una implementación en Python:

\small
\texttt{% Generator: GNU source-highlight, by Lorenzo Bettini, http://www.gnu.org/software/src-highlite
\noindent
\mbox{}\textit{\textcolor{Brown}{\#!/usr/bin/env\ python}} \\
\mbox{}\textit{\textcolor{Brown}{\#\ -*-\ coding:\ utf-8\ -*-}} \\
\mbox{} \\
\mbox{}\textbf{\textcolor{Blue}{def}}\ \textbf{\textcolor{Black}{MultiplicaDV}}\ \textcolor{BrickRed}{(}x\textcolor{BrickRed}{,}y\textcolor{BrickRed}{):} \\
\mbox{}\ \ \ \ signo\ \textcolor{BrickRed}{=}\ \textcolor{Purple}{1}\ \textbf{\textcolor{Blue}{if}}\ x\textcolor{BrickRed}{\textless{}}\textcolor{Purple}{0}\ \textbf{\textcolor{Blue}{and}}\ y\textcolor{BrickRed}{\textless{}}\textcolor{Purple}{0}\ \textbf{\textcolor{Blue}{or}}\ x\textcolor{BrickRed}{\textgreater{}=}\textcolor{Purple}{0}\ \textbf{\textcolor{Blue}{and}}\ y\textcolor{BrickRed}{\textgreater{}=}\textcolor{Purple}{0}\ \textbf{\textcolor{Blue}{else}}\ \textcolor{BrickRed}{-}\textcolor{Purple}{1} \\
\mbox{}\ \ \ \ x\textcolor{BrickRed}{,}y\ \textcolor{BrickRed}{=}\ \textbf{\textcolor{Black}{abs}}\textcolor{BrickRed}{(}x\textcolor{BrickRed}{),}\textbf{\textcolor{Black}{abs}}\textcolor{BrickRed}{(}y\textcolor{BrickRed}{)} \\
\mbox{}\ \ \ \ length\ \textcolor{BrickRed}{=}\ x\textcolor{BrickRed}{.}\textbf{\textcolor{Black}{bit$\_$length}}\textcolor{BrickRed}{()}\ \textbf{\textcolor{Blue}{if}}\ x\textcolor{BrickRed}{.}\textbf{\textcolor{Black}{bit$\_$length}}\textcolor{BrickRed}{()}\ \textcolor{BrickRed}{\textgreater{}}\ y\textcolor{BrickRed}{.}\textbf{\textcolor{Black}{bit$\_$length}}\textcolor{BrickRed}{()}\ \textbf{\textcolor{Blue}{else}}\ y\textcolor{BrickRed}{.}\textbf{\textcolor{Black}{bit$\_$length}}\textcolor{BrickRed}{()} \\
\mbox{} \\
\mbox{}\ \ \ \ \textbf{\textcolor{Blue}{if}}\ \textcolor{BrickRed}{(}length\ \textcolor{BrickRed}{\textless{}}\ \textcolor{Purple}{2}\textcolor{BrickRed}{):} \\
\mbox{}\ \ \ \ \ \ \ \ \textbf{\textcolor{Blue}{return}}\ x\textcolor{BrickRed}{*}y\textcolor{BrickRed}{*}signo \\
\mbox{}\ \ \ \ \textbf{\textcolor{Blue}{else}}\textcolor{BrickRed}{:} \\
\mbox{}\ \ \ \ \ \ \ \ a\ \textcolor{BrickRed}{=}\ x\ \textcolor{BrickRed}{\textgreater{}\textgreater{}}\ length\textcolor{BrickRed}{/}\textcolor{Purple}{2} \\
\mbox{}\ \ \ \ \ \ \ \ b\ \textcolor{BrickRed}{=}\ x\ \textcolor{BrickRed}{\%}\ \textcolor{BrickRed}{(}\textcolor{Purple}{1}\ \textcolor{BrickRed}{\textless{}\textless{}}\ length\textcolor{BrickRed}{/}\textcolor{Purple}{2}\textcolor{BrickRed}{)} \\
\mbox{}\ \ \ \ \ \ \ \ c\ \textcolor{BrickRed}{=}\ y\ \textcolor{BrickRed}{\textgreater{}\textgreater{}}\ length\textcolor{BrickRed}{/}\textcolor{Purple}{2} \\
\mbox{}\ \ \ \ \ \ \ \ d\ \textcolor{BrickRed}{=}\ y\ \textcolor{BrickRed}{\%}\ \textcolor{BrickRed}{(}\textcolor{Purple}{1}\ \textcolor{BrickRed}{\textless{}\textless{}}\ length\textcolor{BrickRed}{/}\textcolor{Purple}{2}\textcolor{BrickRed}{)}\ \  \\
\mbox{} \\
\mbox{}\ \ \ \ \ \ \ \ ac\ \textcolor{BrickRed}{=}\ \textbf{\textcolor{Black}{MultiplicaDV}}\ \textcolor{BrickRed}{(}a\textcolor{BrickRed}{,}c\textcolor{BrickRed}{)} \\
\mbox{}\ \ \ \ \ \ \ \ bd\ \textcolor{BrickRed}{=}\ \textbf{\textcolor{Black}{MultiplicaDV}}\ \textcolor{BrickRed}{(}b\textcolor{BrickRed}{,}d\textcolor{BrickRed}{)} \\
\mbox{} \\
\mbox{}\ \ \ \ \ \ \ \ \textbf{\textcolor{Blue}{return}}\ \ \textcolor{BrickRed}{((}ac\ \textcolor{BrickRed}{\textless{}\textless{}}\ \textcolor{BrickRed}{(}length\textcolor{BrickRed}{/}\textcolor{Purple}{2}\textcolor{BrickRed}{*}\textcolor{Purple}{2}\textcolor{BrickRed}{))}\ \textcolor{BrickRed}{+}\  \\
\mbox{}\ \ \ \ \ \ \ \ \ \ \ \ \ \ \ \ \textcolor{BrickRed}{((}\textbf{\textcolor{Black}{MultiplicaDV}}\textcolor{BrickRed}{(}a\textcolor{BrickRed}{-}b\textcolor{BrickRed}{,}d\textcolor{BrickRed}{-}c\textcolor{BrickRed}{)}\ \textcolor{BrickRed}{+}\ ac\ \textcolor{BrickRed}{+}\ bd\textcolor{BrickRed}{)}\ \textcolor{BrickRed}{\textless{}\textless{}}\ length\textcolor{BrickRed}{/}\textcolor{Purple}{2}\textcolor{BrickRed}{)}\ \textcolor{BrickRed}{+} \\
\mbox{}\ \ \ \ \ \ \ \ \ \ \ \ \ \ \ \ bd\textcolor{BrickRed}{)*}signo \\
\mbox{}\ \ \ \  \\
\mbox{} \\
\mbox{}x\ \textcolor{BrickRed}{=}\ \textbf{\textcolor{Black}{input}}\ \textcolor{BrickRed}{(}\texttt{\textcolor{Red}{"{}Primer\ número:\ "{}}}\textcolor{BrickRed}{)} \\
\mbox{}y\ \textcolor{BrickRed}{=}\ \textbf{\textcolor{Black}{input}}\ \textcolor{BrickRed}{(}\texttt{\textcolor{Red}{"{}Segundo\ número:\ "{}}}\textcolor{BrickRed}{)} \\
\mbox{}\textit{\textcolor{Brown}{\#x\ =\ 0b10011011}} \\
\mbox{}\textit{\textcolor{Brown}{\#y\ =\ 0b10111010}} \\
\mbox{} \\
\mbox{}\textbf{\textcolor{Blue}{print}}\ \texttt{\textcolor{Red}{"{}Resultado:\ "{}}}\textcolor{BrickRed}{,}\ \textbf{\textcolor{Black}{MultiplicaDV}}\textcolor{BrickRed}{(}x\textcolor{BrickRed}{,}y\textcolor{BrickRed}{)}
}
\normalsize

\section {$K$-ésimo menor elemento de un vector}
  \subsection{Algoritmo}
    Nos basaremos en una idea similar a la usada para el algoritmo de ordenación \textit{Quicksort}.
    Seleccionaremos un elemento pivote y dividiremos la lista en dos partes, una con los elementos menores
    a él y otra con los elementos mayores que él.
    
    Así, el pivote quedará en su posición ordenada. Si ésta es
    la posición $k$, habremos acabado; en otro caso, sólo tendremos que buscar recursivamente sobre la 
    sublista de elementos mayores, si el índice es menor que $k$; o sobre la sublista de elementos menores, si el
    índice es mayor que $k$.
    
    \begin{algorithm}[H]
	\begin{algorithmic}[1]
		  \REQUIRE \ \\
		  \texttt{v}, vector de números\\
		  \texttt{k}, entero postivo\\\
	  \STATE{\texttt{pivote$\leftarrow$ v[0]}}     	
	  \STATE{Definimos:\\
		\texttt{menores}=$\{x\in v : x<pivote\}$\\
		\texttt{iguales}=$\{x\in v : x=pivote\}$\\
		\texttt{mayores}=$\{x\in v : x>pivote\}$}
	  \\\
	  \IF{\texttt{k<\#menores}}
	    \RETURN {\texttt{k-ésimo(menores,k)}}
	  \ELSIF{\texttt{k<(\#menores+\#iguales)}}
	    \RETURN {pivote}
	  \ELSE
	    \RETURN {\texttt{k-ésimo(mayores,k-(\#menores+\#iguales))}}
	  \ENDIF
	\end{algorithmic}
      \caption{$k$-ésimo menor elemento de un vector. Primera versión.}
      \label{kesimo}
    \end{algorithm}

    Nótese como la elección del pivote en el algoritmo original es arbitraria. Esta elección condiciona la eficiencia de
    todo el algoritmo. El pivote óptimo para el algoritmo sería la mediana de los elementos de la lista, pero su determinación
    es costosa. En lugar de usar la mediana, buscamos un elemento que se aproxime lo suficiente a la mediana para asegurarnos
    la mejora de eficiencia en el algoritmo.
    
    Se propone como algoritmo de obtención del pivote dividir la lista en sublistas de $5$ elementos, de las que se obtendrá
    la mediana usando cualquier algoritmo de ordenación. Sobre las medianas obtenidas, se aplica recursivamente el algoritmo.
    Se obtiene así una pseudomediana suficientemente cercana a la verdadera mediana de la lista.
    \begin{algorithm}[H]
      \begin{algorithmic}
	    \REQUIRE \ \\
	    \texttt{v}, vector de números \\\
	  
	  \IF{\texttt{v.size >\ 5}}
	    % (Comentario Mario)
	    % Yo sé que escribir 5\lfloor \frac{n}{5} \rfloor parece más confuso, pero no podemos ni debemos suponer que n es múltiplo de 5.
	    % Escribir (n-4) no es correcto. De todas formas, habría que buscar una forma de escribirlo que fuera consistente.
	    \STATE{\texttt{sublistas} $\leftarrow$ $[v[0..4],v[5..9],\dots,v[(n-4)..n]]$} \\
	    \STATE{\texttt{w} $\leftarrow$ \{ \texttt{pivote($s_i$)} $\ | \ $ $s_i$ $\in$ \texttt{sublistas} \} } \\
	    \RETURN{\texttt{pivote(w)}}
	  \ELSE
	    \STATE{Ordenar \texttt{v}} \\
	    \RETURN{\texttt{v[2]}} \\
	  \ENDIF
      \end{algorithmic}
      \caption{Selección del pivote.}
      \label{median-of-medians}
    \end{algorithm}

  \subsection{Análisis de eficiencia.}
    \subsubsection{Elección arbitraria del pivote.}
      El caso medio del algoritmo es lineal. Puede comprobarse siguiendo un razonamiento similar al aplicado para 
      calcular el caso medio en \textit{Quicksort}, podemos analizar su eficiencia.
    
      El algoritmo realiza una partición alrededor del pivote en tiempo $\mathcal{O}(n)$ y pasa a aplicar el algoritmo
      en uno de los dos lados de la lista dividida. Cada uno tendrá tanta posibilidad de ser elegido como elementos tenga,
      por lo que podemos considerar:
      \begin{equation}
	T(n) = n + \frac{1}{n}\sum_{i=0}^{n-1}\left( \frac{i}{n}T(i) + \frac{n-1-i}{n}T(n-1-i) \right)
      \end{equation}
      Reordenando y multiplicando por $n^2$ obtenemos:
      \begin{equation}
        n^2T(n) = n^3 + 2\sum^{n-1}_{i=1} \left( i T(i) \right)
      \end{equation}
      Para resolver la ecuación no homogénea, evaluamos en $n-1$ y restamos, obteniendo:
      \begin{equation}
        n^2T(n) = (n^2-1)T(n-1) + 3n^2 - 3n + 1
      \end{equation}
      Ahora consideramos $S(n) = \frac{nT(n)}{n+1}$, y obtenemos la ecuación:
      \begin{equation}
        S(n) = S(n-1) + 3\frac{n-1}{n+1} + \frac{1}{n} - \frac{1}{n-1} \\
      \end{equation}
      Que se puede resolver inductivamente usando la serie armónica.
      \begin{eqnarray}
        \lefteqn{S(n) = \sum_{i=2}^n{\left( 3 - \frac{6}{i+1} + \frac{1}{i} - \frac{1}{i-1} \right)} = }\\ 
        &&\mathcal{O}(n) - \mathcal{O}(log(n)) + \mathcal{O}(log(n)) - \mathcal{O}(log(n)) = \mathcal{O}(n)
      \end{eqnarray}
      De donde, por ser $\frac{n+1}{n}\longrightarrow 1$ despejamos de nuevo $T(n)$ como:
      \begin{equation}
        T(n) =  \mathcal{O}(n)
      \end{equation}

      Sin embargo, el caso peor con elección arbitraria es $\mathcal{O}(n^2)$. Trivialmente, escogiendo a cada paso el mínimo de los elementos restantes como pivote
      mientras que buscamos el elemento máximo de la lista, nos lleva a aplicar la partición $n$ veces, cada una de ellas con coste $\mathcal{O}(n)$.
    
    \subsubsection{Elección de la pseudomediana}
      El elegir la mediana siguiendo el algoritmo propuesto anteriormente permite escoger un pivote que sea lo suficientemente cercano a la verdadera mediana.
      Acotmos el número de elementos menores que la pseudomediana. Sabemos que una mediana así elegida nos permite afirmar que, de los elementos de las sublistas
      con medianas menores al pivote, hasta $\frac{5}{2} \lceil \frac{n}{5} \rceil$ pueden ser menores. De los elementos de las sublistas con medianas mayores,
      sólo 2 de ellos podrían ser menores, dando en total $ \lceil \frac{n}{5} \rceil$ elementos menores. Añadimos además los dos elementos de la sublista menores a la pseudomediana.
      
      Tenemos entonces que el número de elementos menores queda acotado por:
      \begin{equation}
       \frac{5}{2} \left\lceil \frac{n}{5} \right\rceil + \left\lceil \frac{n}{5} \right\rceil + 2 \leq \frac{7n}{10} + \mathcal{O}(1) 
      \end{equation}
      Lo que asegura que la pseudomediana está entre aproximadamente el percentil $70$ de la lista.
      
      El algoritmo satisface entonces la siguiente ecuación en recurrencia, donde los sumandos se corresponden con la aplicación
      recursiva del algoritmo, la aplicación sobre la lista que queda tras la pseudomediana y el proceso de pivotar:
      \begin{equation}
       T(n) \leq T\left(\left\lceil \frac{n}{5} \right\rceil \right) + T\left(\frac{7n}{10} + \mathcal{O}(1) \right) + \mathcal{O}(n)
      \end{equation}
      Lo que nos lleva, por inducción suponiendo $T(n) = cn$, a que el algoritmo es lineal.
      
\subsection{Implementación}
A continuación mostramos una implementación en el lenguaje Ruby:

\small
\texttt{% Generator: GNU source-highlight, by Lorenzo Bettini, http://www.gnu.org/software/src-highlite
\noindent
\mbox{}\textit{\textcolor{Brown}{\#!/usr/bin/env\ ruby}} \\
\mbox{}\textit{\textcolor{Brown}{\#\ -*-\ coding:\ utf-8\ -*-}} \\
\mbox{}\textit{\textcolor{Brown}{\#encoding\ utf8}} \\
\mbox{} \\
\mbox{}\textit{\textcolor{Brown}{\#\ Creo\ que\ es\ eficiencia\ O(n**2)\ en\ el\ peor\ caso\ y\ O(n)\ en\ el\ medio.}} \\
\mbox{} \\
\mbox{}\textbf{\textcolor{Blue}{class}}\ Array \\
\mbox{}\ \ \textbf{\textcolor{Blue}{def}}\ kesimo\ \textcolor{BrickRed}{(}k\textcolor{BrickRed}{)} \\
\mbox{}\ \ \ \ \textit{\textcolor{Brown}{\#\ Esto\ tiene\ eficiencia\ O(n).}} \\
\mbox{}\ \ \ \ \textit{\textcolor{Brown}{\#\ Partimos\ la\ lista\ en\ dos,\ sobre\ un\ pivote.}} \\
\mbox{}\ \ \ \ pivote\ \ \textcolor{BrickRed}{=}\ \textbf{\textcolor{Blue}{self}}\textcolor{BrickRed}{.}first \\
\mbox{}\ \ \ \ menores\ \textcolor{BrickRed}{=}\ \textbf{\textcolor{Blue}{self}}\textcolor{BrickRed}{.}select\ \textcolor{Red}{\{}\ \textcolor{BrickRed}{$|$}x\textcolor{BrickRed}{$|$}\ x\ \textcolor{BrickRed}{\textless{}}\ pivote\ \textcolor{Red}{\}} \\
\mbox{}\ \ \ \ mayores\ \textcolor{BrickRed}{=}\ \textbf{\textcolor{Blue}{self}}\textcolor{BrickRed}{.}select\ \textcolor{Red}{\{}\ \textcolor{BrickRed}{$|$}x\textcolor{BrickRed}{$|$}\ x\ \textcolor{BrickRed}{\textgreater{}}\ pivote\ \textcolor{Red}{\}} \\
\mbox{}\ \ \ \ iguales\ \textcolor{BrickRed}{=}\ \textbf{\textcolor{Blue}{self}}\textcolor{BrickRed}{.}select\ \textcolor{Red}{\{}\ \textcolor{BrickRed}{$|$}x\textcolor{BrickRed}{$|$}\ x\ \textcolor{BrickRed}{==}\ pivote\ \textcolor{Red}{\}} \\
\mbox{} \\
\mbox{}\ \ \ \ \textit{\textcolor{Brown}{\#\ Estudiamos\ si\ el\ pivote\ es\ el\ elemento\ buscado.}} \\
\mbox{}\ \ \ \ \textit{\textcolor{Brown}{\#\ Buscamos\ en\ el\ subarray\ correspondiente.}} \\
\mbox{}\ \ \ \ particion$\_$menores\ \textcolor{BrickRed}{=}\ menores\textcolor{BrickRed}{.}size \\
\mbox{}\ \ \ \ particion$\_$iguales\ \textcolor{BrickRed}{=}\ menores\textcolor{BrickRed}{.}size\ \textcolor{BrickRed}{+}\ iguales\textcolor{BrickRed}{.}size \\
\mbox{} \\
\mbox{}\ \ \ \ \textbf{\textcolor{Blue}{if}}\ k\ \textcolor{BrickRed}{\textless{}}\ particion$\_$menores \\
\mbox{}\ \ \ \ \ \ \textbf{\textcolor{Blue}{return}}\ menores\textcolor{BrickRed}{.}kesimo\ \textcolor{BrickRed}{(}k\textcolor{BrickRed}{)} \\
\mbox{}\ \ \ \ \textbf{\textcolor{Blue}{elsif}}\ k\ \textcolor{BrickRed}{\textless{}}\ particion$\_$iguales \\
\mbox{}\ \ \ \ \ \ \textbf{\textcolor{Blue}{return}}\ pivote \\
\mbox{}\ \ \ \ \textbf{\textcolor{Blue}{else}} \\
\mbox{}\ \ \ \ \ \ \textbf{\textcolor{Blue}{return}}\ mayores\textcolor{BrickRed}{.}kesimo\ \textcolor{BrickRed}{(}k\ \textcolor{BrickRed}{-}\ particion$\_$iguales\textcolor{BrickRed}{)} \\
\mbox{}\ \ \ \ \textbf{\textcolor{Blue}{end}} \\
\mbox{}\ \ \textbf{\textcolor{Blue}{end}} \\
\mbox{}\textbf{\textcolor{Blue}{end}} \\
\mbox{} \\
\mbox{} \\
\mbox{}\textbf{\textcolor{Blue}{if}}\ \textbf{\textcolor{Blue}{$\_$$\_$FILE$\_$$\_$}}\ \textcolor{BrickRed}{==}\ \textcolor{ForestGreen}{\$0} \\
\mbox{}\ \ puts\ \texttt{\textcolor{Red}{"{}Introduce\ array:\ "{}}} \\
\mbox{}\ \ line\ \textcolor{BrickRed}{=}\ gets\textcolor{BrickRed}{.}chomp \\
\mbox{}\ \ array\ \textcolor{BrickRed}{=}\ line\textcolor{BrickRed}{.}split\textcolor{BrickRed}{.}map\textcolor{BrickRed}{(\&:}to$\_$i\textcolor{BrickRed}{)} \\
\mbox{} \\
\mbox{}\ \ puts\ \texttt{\textcolor{Red}{"{}Introduce\ k:\ "{}}} \\
\mbox{}\ \ k\ \textcolor{BrickRed}{=}\ \textcolor{BrickRed}{(}gets\textcolor{BrickRed}{.}chomp\textcolor{BrickRed}{).}to$\_$i \\
\mbox{}\ \ puts\ \texttt{\textcolor{Red}{"{}k-ésimo\ elemento:\ \#\{array.kesimo(k)\}."{}}} \\
\mbox{}\textbf{\textcolor{Blue}{end}} \\
\mbox{}
}
\normalsize

y una implementación en Haskell:

\small
\texttt{% Generator: GNU source-highlight, by Lorenzo Bettini, http://www.gnu.org/software/src-highlite
\noindent
\mbox{}\textbf{\textcolor{Blue}{import}}\ Data\textcolor{BrickRed}{.}\textcolor{ForestGreen}{List} \\
\mbox{}\textbf{\textcolor{Blue}{import}}\ Data\textcolor{BrickRed}{.}List\textcolor{BrickRed}{.}\textcolor{ForestGreen}{Split} \\
\mbox{} \\
\mbox{}\textit{\textcolor{Brown}{\{-}} \\
\mbox{}\textit{\textcolor{Brown}{\ \ Caso\ medio\ lineal,\ caso\ peor\ cuadrático.}} \\
\mbox{}\textit{\textcolor{Brown}{\ \ Toma\ un\ pivote,\ parte\ la\ lista\ en\ tres\ partes\ y\ toma\ la\ parte\ del\ késimo\ elemento.\ }} \\
\mbox{}\textit{\textcolor{Brown}{\ -\}}} \\
\mbox{}kesimo\ \textcolor{BrickRed}{::}\ \textcolor{BrickRed}{(}\textcolor{ForestGreen}{Ord}\ a\textcolor{BrickRed}{)}\ \textcolor{BrickRed}{=\textgreater{}}\ \textcolor{BrickRed}{[}a\textcolor{BrickRed}{]}\ \textcolor{BrickRed}{-\textgreater{}}\ \textcolor{ForestGreen}{Int}\ \textcolor{BrickRed}{-\textgreater{}}\ \textcolor{ForestGreen}{Maybe}\ a \\
\mbox{}kesimo\ \textcolor{BrickRed}{[]}\ \textbf{\textcolor{Blue}{$\_$}}\ \textcolor{BrickRed}{=}\ \textcolor{ForestGreen}{Nothing} \\
\mbox{}kesimo\ xs\ k \\
\mbox{}\ \ \textcolor{BrickRed}{$|$}\ k\ \textcolor{BrickRed}{\textless{}}\ parte1\ \textcolor{BrickRed}{=}\ kesimo\ menores\ k \\
\mbox{}\ \ \textcolor{BrickRed}{$|$}\ k\ \textcolor{BrickRed}{\textless{}}\ parte2\ \textcolor{BrickRed}{=}\ \textcolor{ForestGreen}{Just}\ pivote\  \\
\mbox{}\ \ \textcolor{BrickRed}{$|$}\ otherwise\ \ \textcolor{BrickRed}{=}\ kesimo\ mayores\ \textcolor{BrickRed}{(}k\textcolor{BrickRed}{-}parte2\textcolor{BrickRed}{)} \\
\mbox{}\ \ \ \ \textbf{\textcolor{Blue}{where}}\ menores\ \textcolor{BrickRed}{=}\ filter\ \textcolor{BrickRed}{(\textless{}}\ pivote\textcolor{BrickRed}{)}\ xs \\
\mbox{}\ \ \ \ \ \ \ \ \ \ mayores\ \textcolor{BrickRed}{=}\ filter\ \textcolor{BrickRed}{(\textgreater{}}\ pivote\textcolor{BrickRed}{)}\ xs \\
\mbox{}\ \ \ \ \ \ \ \ \ \ iguales\ \textcolor{BrickRed}{=}\ filter\ \textcolor{BrickRed}{(==}pivote\textcolor{BrickRed}{)}\ xs \\
\mbox{}\ \ \ \ \ \ \ \ \ \ parte1\ \ \textcolor{BrickRed}{=}\ length\ menores \\
\mbox{}\ \ \ \ \ \ \ \ \ \ parte2\ \ \textcolor{BrickRed}{=}\ parte1\ \textcolor{BrickRed}{+}\ \textcolor{BrickRed}{(}length\ iguales\textcolor{BrickRed}{)} \\
\mbox{}\ \ \ \ \ \ \ \ \ \ pivote\ \ \textcolor{BrickRed}{=}\ medianOfMedians\ xs \\
\mbox{} \\
\mbox{}\textit{\textcolor{Brown}{\{-}} \\
\mbox{}\textit{\textcolor{Brown}{\ \ Elección\ de\ la\ pseudomediana.}} \\
\mbox{}\textit{\textcolor{Brown}{\ \ Partimos\ la\ lista\ en\ sublistas\ de\ 5\ elementos,\ obtenemos\ la\ mediana\ y\ formamos\ nueva\ lista.}} \\
\mbox{}\textit{\textcolor{Brown}{\ -\}}} \\
\mbox{}medianOfMedians\ \textcolor{BrickRed}{::}\ \textcolor{BrickRed}{(}\textcolor{ForestGreen}{Ord}\ a\textcolor{BrickRed}{)}\ \textcolor{BrickRed}{=\textgreater{}}\ \textcolor{BrickRed}{[}a\textcolor{BrickRed}{]}\ \textcolor{BrickRed}{-\textgreater{}}\ a \\
\mbox{}medianOfMedians\ \textcolor{BrickRed}{[}x\textcolor{BrickRed}{]}\ \textcolor{BrickRed}{=}\ x \\
\mbox{}medianOfMedians\ xs\ \ \textcolor{BrickRed}{=}\ medianOfMedians\ \textcolor{BrickRed}{\$}\ map\ median\ \textcolor{BrickRed}{(}chunksOf\ \textcolor{Purple}{5}\ xs\textcolor{BrickRed}{)} \\
\mbox{}\ \ \textbf{\textcolor{Blue}{where}}\ median\ xs\ \textcolor{BrickRed}{=}\ \textcolor{BrickRed}{(}sort\ xs\textcolor{BrickRed}{)}\ \textcolor{BrickRed}{!!}\ \textcolor{BrickRed}{(}div\ \textcolor{BrickRed}{(}length\ xs\textcolor{BrickRed}{)}\ \textcolor{Purple}{2}\textcolor{BrickRed}{)}
}
\normalsize


\section {Elemento mayoritario}
\subsection{Algoritmo}
Se denomina \textit{elemento mayoritario} de una lista a aquel que aparece más de $n/2$ veces, donde $n$ es el tamaño de la lista.
El problema reside en eliminar la necesidad de contar todas las ocurrencias de cada uno de los elementos, de forma que podamos conocer el mayoritario más eficientemente.
Para ello, utilizaremos la siguiente estrategia:

\begin{itemize}
\item Tomamos $\left\lfloor\frac{n}{2}\right\rfloor$ parejas de elementos consecutivos en la lista.
\item Para cada pareja, comprobaremos cuáles tienen sus dos elementos iguales.
\item Creamos una nueva lista con los elementos correspondientes a cada pareja de valores repetidos.
\end{itemize}

Efectivamente, se justifica mediante el principio del palomar que habrá al menos una pareja que contenga al elemento mayoritario repetido (excepto en el caso de tamaño impar y elemento mayoritario en la última posición, donde podría no haber parejas de elementos repetidos), y podemos ver que cada pareja de elementos repetidos distintos del mayoritario provocará la existencia de una pareja con el mayoritario.

Una vez encontrado el candidato a elemento mayoritario mediante el algoritmo, es necesario comprobar si realmente lo es, ya que la condición que tenemos es necesaria pero no suficiente. Además, si el tamaño del vector es impar, la estrategia no considerará el último elemento (ya que no entra en ninguna pareja), luego habrá que comprobarlo después. Estas operaciones se realizan trivialmente en $\mathcal{O}(n)$.

\begin{algorithm}[H]
	\begin{algorithmic}[1]
		\REQUIRE \ \\
        	\texttt{elem}, vector de elementos\\\
     	\IF{\texttt{\#elem} <\ 3}
	     	\IF{\texttt{\#elem} <\ 2 \OR \texttt{elem[0]} == \texttt{elem[1]}}
    	 		\RETURN{\texttt{elem[0]}} 
     		\ELSE
     			\RETURN{\texttt{NULO}}
     		\ENDIF \\\
     		
     		\STATE{Crear un nuevo vector \texttt{repet}}
     		\FORALL{\texttt{(p, q)} pareja consecutiva \textbf{en} \texttt{elem}}
	     		\IF{\texttt{p} == \texttt{q}}
		     		\STATE{Añadir \texttt{p} a \texttt{repet}}
	     		\ENDIF
     		\ENDFOR
    	\ENDIF \\\
    	\STATE{\texttt{may} $\gets$ \texttt{mayoritario(repet)}}
    	\IF{(ocurrencias de \texttt{may}) >\ \#\texttt{elem}/2}
     		\RETURN{\texttt{may}}
     	\ELSE
     		\IF{(ocurrencias de \texttt{elem[\#elem - 1]}) >\ \#\texttt{elem}/2}
     			\RETURN{\texttt{elem[\#elem - 1]}}
     		\ELSE
     			\RETURN{\texttt{NULO}}
     		\ENDIF
    	\ENDIF
	\end{algorithmic}
    \caption{Búsqueda del elemento mayoritario}
    \label{mayoritario}
\end{algorithm}

\subsection{Análisis de eficiencia}
El algoritmo produce una única llamada recursiva con un argumento de tamaño $\frac{n}{2}$ en el peor caso, siendo $n$ el tamaño actual. Por tanto tenemos la recurrencia
\begin{equation}
\label{eqmay}
T(n)=\left\lbrace
	    \begin{array}{lr}
            1 & n\le 2\\
            T\left(\frac{n}{2}\right) + n & n>2\\
            \end{array}
	    \right.
\end{equation} 
Cambiamos de variable, $m = log_2(n);\ n = 2^m$, y resolvemos la ecuación homogénea asociada:
$$ x_m-x_{m-1}=0 \Rightarrow x_m = 1^m\ c_1 $$
Y hallamos una solución particular para la ecuación en recurrencia no homogénea. Por ejemplo, la sucesión $\{0, 2, 6, 14, 30\dots\}=\{2^{m+1}-2\}_{m\ge 0}$ cumple la ecuación. La solución general es por tanto
$$ x_m = 1^m\ c_1 + 2^{m+1}-2\ \forall m\ge 0$$
Al deshacer el cambio de variable tenemos que el algoritmo de búsqueda del elemento mayoritario es lineal:
$$ T(n) = c_1 + 2n-2\ \in \mathcal{O}(n)$$

% % %
% No sé si añadir todas las implementaciones al final
% o para cada algoritmo, por ahora la he puesto aquí.
% Cambiadla si lo veis conveniente
% -David
% % %
\subsection{Implementación}
A continuación mostramos una implementación en el lenguaje Ruby:

\small
\texttt{% Generator: GNU source-highlight, by Lorenzo Bettini, http://www.gnu.org/software/src-highlite
\noindent
\mbox{}\textit{\textcolor{Brown}{\#!/usr/bin/env\ ruby}} \\
\mbox{}\textit{\textcolor{Brown}{\#encoding\ utf8}} \\
\mbox{} \\
\mbox{}\textit{\textcolor{Brown}{\#\ Búsqueda\ del\ elemento\ mayoritario.}} \\
\mbox{}\textit{\textcolor{Brown}{\#\ Eficiencia:\ O(n)}} \\
\mbox{} \\
\mbox{}\textbf{\textcolor{Blue}{class}}\ Array \\
\mbox{}\ \ \textbf{\textcolor{Blue}{def}}\ mayoritario \\
\mbox{}\ \ \ \ \textit{\textcolor{Brown}{\#\ Caso\ base}} \\
\mbox{}\ \ \ \ \textbf{\textcolor{Blue}{return}}\ \textcolor{BrickRed}{(}first\ \textcolor{BrickRed}{==}\ last\ \textcolor{BrickRed}{?}\ first\ \textcolor{BrickRed}{:}\ \textbf{\textcolor{Blue}{nil}}\textcolor{BrickRed}{)}\ \textbf{\textcolor{Blue}{if}}\ size\ \textcolor{BrickRed}{\textless{}}\ \textcolor{Purple}{3} \\
\mbox{}\ \ \ \  \\
\mbox{}\ \ \ \ pares$\_$repet\ \textcolor{BrickRed}{=}\ \textcolor{BrickRed}{[]} \\
\mbox{} \\
\mbox{}\ \ \ \ \textit{\textcolor{Brown}{\#\ Tomamos\ las\ parejas\ de\ elementos\ repetidos}} \\
\mbox{}\ \ \ \ each$\_$slice\textcolor{BrickRed}{(}\textcolor{Purple}{2}\textcolor{BrickRed}{)}\ \textcolor{Red}{\{}\ \textcolor{BrickRed}{$|$}e\textcolor{BrickRed}{$|$} \\
\mbox{}\ \ \ \ \ \ pares$\_$repet\ \textcolor{BrickRed}{\textless{}\textless{}}\ e\textcolor{BrickRed}{[}\textcolor{Purple}{0}\textcolor{BrickRed}{]}\ \textbf{\textcolor{Blue}{if}}\ e\textcolor{BrickRed}{[}\textcolor{Purple}{0}\textcolor{BrickRed}{]}\ \textcolor{BrickRed}{==}\ e\textcolor{BrickRed}{[}\textcolor{Purple}{1}\textcolor{BrickRed}{]} \\
\mbox{}\ \ \ \ \textcolor{Red}{\}} \\
\mbox{} \\
\mbox{}\ \ \ \ \textit{\textcolor{Brown}{\#\ pares$\_$repet.size\ \textless{}=\ size/2,\ por\ lo\ que}} \\
\mbox{}\ \ \ \ \textit{\textcolor{Brown}{\#\ la\ llamada\ recursiva\ es\ como\ mucho\ T(n/2)}} \\
\mbox{}\ \ \ \ posible\ \textcolor{BrickRed}{=}\ pares$\_$repet\textcolor{BrickRed}{.}mayoritario \\
\mbox{} \\
\mbox{}\ \ \ \ \textit{\textcolor{Brown}{\#\ Cuenta\ de\ ocurrencias\ es\ O(n)}} \\
\mbox{}\ \ \ \ \textbf{\textcolor{Blue}{return}}\ posible\ \textbf{\textcolor{Blue}{if}}\ count\textcolor{BrickRed}{(}posible\textcolor{BrickRed}{)}\ \textcolor{BrickRed}{\textgreater{}}\ size\textcolor{BrickRed}{/}\textcolor{Purple}{2} \\
\mbox{}\ \ \ \ \textbf{\textcolor{Blue}{return}}\ last\ \textbf{\textcolor{Blue}{if}}\ size\ \textcolor{BrickRed}{\%}\ \textcolor{Purple}{2}\ \textcolor{BrickRed}{==}\ \textcolor{Purple}{1}\ \textbf{\textcolor{Blue}{and}}\ count\textcolor{BrickRed}{(}last\textcolor{BrickRed}{)}\ \textcolor{BrickRed}{\textgreater{}}\ size\textcolor{BrickRed}{/}\textcolor{Purple}{2} \\
\mbox{}\ \ \textbf{\textcolor{Blue}{end}} \\
\mbox{}\textbf{\textcolor{Blue}{end}} \\
\mbox{} \\}
\normalsize

\section {Tornillos y tuercas}
  \subsection{Algoritmo}

    Tenemos dos listas cuyos elementos están emparejados con uno y sólo uno de los elementos de la otra lista.
    
    De nuevo, vamos a utilizar un procedimiento similar al usado por el \textit{Quicksort}.
    
    Seleccionaremos un elemento pivote de una lista y dividiremos la lista contraria en tres partes, una con los elementos menores a él, otra con los mayores, y su pareja.
    
    Y utilizando la pareja como pivote, dividimos la primera lista en elementos menores y mayores a la pareja(y, por tanto, al elemento que cogimos antes).
    
    Una vez tenemos la separación en dos partes de las dos listas, procedemos a ordenar cada una de las mitades con la mitad correspondiente de la otra lista(elementos menores al pivote con elementos menores a su pareja, y elementos mayores al pivote con elementos mayores a su pareja).
\begin{algorithm}[H]
	\begin{algorithmic}[1]
	  \REQUIRE \ \\
	    \texttt{a, b, listas emparejadas de $n$ números}\\
	  \IF{$n$==0}
	    \RETURN {\texttt{[]}}
	  \ELSE
	    \STATE{\texttt{pivote$\leftarrow$ a[0]}}
	    \STATE{Definimos:\\
	      \texttt{pareja}=$\{y\in b : y=pivote\}$, que será único por las condiciones del problema\\
	      \texttt{menores de b}=$\{y\in b : y<pivote\}$\\
	      \texttt{mayores de b}=$\{y\in b : y>pivote\}$\\
	      \texttt{menores de a}=$\{x\in a : x<pareja\}$\\
	      \texttt{mayores de a}=$\{x\in a : x>pareja\}$\\}
	  \IF{\#\texttt{menores de b} != 0}
	    \STATE{\texttt{Ordenar(menores de b, menores de a)}}
	  \ENDIF \\
	  \IF{\#\texttt{mayores de b} != 0}
	    \STATE{\texttt{Ordenar(mayores de b, mayores de a)}}
	  \ENDIF \\
	  
	  \RETURN {\texttt{[(menores de a + pivote + mayores de a), \\ (menores de b + pareja + mayores de b)]}}
	  \ENDIF \\\
	\end{algorithmic}
    \caption{Tornillos y tuercas}
    \label{TyT}
\end{algorithm}
  \subsection{Análisis de eficiencia.}
      Dada la gran similitud de el algoritmo usado con el algoritmo \textit{Quicksort}, la eficiencia también será similar.
      
      En el peor de los casos, tenemos que el pivote(y su pareja) quedan al principio o al final de sus correspondientes listas, dejando una lista vacía y una lista de tamaño $n-1$. En este caso, tenemos una eficiencia cuadrática:
      
	  \begin{equation}
	  	T(n) = 2n + T(0) + T(n-1) = 2n + 1 + T(n-1) = \dots = 2n^2 + n - 1
	  \end{equation}
      
      Por otra parte, el mejor caso consigue dividir la lista por la mitad. En este caso, nos resuelve el problema con eficiencia $\mathcal{O}(nlog(n))$:
      
	  \begin{eqnarray*}
	  \lefteqn{T(n) = 2n + T\bigg(\left\lceil \frac{n-1}{2} \right\rceil\bigg) + T\bigg(\left\lfloor \frac{n-1}{2}\right\rfloor\bigg) \approx 2n + 2T\bigg(\frac{n}{2}\bigg) \approx} \\
	  && \approx 2n + 2n + 4T\bigg(\frac{n}{4}\bigg) = 4n + 4T\bigg(\frac{n}{4}\bigg)\approx 4n + 2n + 8T\bigg(\frac{n}{8}\bigg) = \\ 
	  && = 6n + 8T\bigg(\frac{n}{8}\bigg) \approx \dots \approx 2nlog_2(n) + n T(0) = 2nlog_2(n) + n
	  \end{eqnarray*}
	  
      El caso medio del algoritmo es $\mathcal{O}(nlog(n)) $. Puede comprobarse siguiendo un razonamiento similar al aplicado para calcular el caso medio en \textit{Quicksort}, podemos analizar su eficiencia.
    
      El algoritmo realiza dos particiones, una alrededor del pivote y otra alrededor de su pareja, ambas en tiempo $\mathcal{O}(n)$. Ahora, aplicamos el algoritmo
      en cada una de las dos parejas de listas asociadas resultantes. Cada elemento tendrá la misma posibilidad de ser elegido que los demás, es decir, $\frac{1}{n}$,
      con $n$, el número de elementos de las listas, por lo que podemos considerar:
      \begin{equation}
		T(n) = 2n + \frac{1}{n}\sum_{i=0}^{n-1}\left(T(i) + T(n-1-i) \right)
      \end{equation}
      
      Al evaluarlo en $n-1$, queda:
      \begin{equation}
      	T(n-1) = 2(n-1) + \frac{1}{n-1}\sum_{i=0}^{n-2}\left(T(i) + T((n-2-i) \right)
      \end{equation}
      Sumando las ecuaciones anteriores multiplicando por $n$ la primera ecuación y por $-(n-1)$ la otra, obtenemos:
      \begin{equation}
        nT(n) - (n-1)T(n-1) = 2n^2 + 2\sum^{n-1}_{i=1} T(i) - 2(n-1)^2 - 2\sum_{i=0}^{n-2}T(i)
      \end{equation}
      Simplificando, nos quedaría:
	  \begin{equation}
	    nT(n) - (n-1)T(n-1) = 4n-2 + 2(T(n-1) - T(0)) = 4n + 2T(n-1) 
	  \end{equation}
	  \begin{equation}
	    \frac{T(n)}{n+1} = \frac{T(n-1)}{n} + 4
	  \end{equation}
      Ahora consideramos $S(n) = \frac{T(n)}{n+1}$, y obtenemos la ecuación:
      \begin{equation}
        S(n) = S(n-1) + \frac{4}{n+1}
      \end{equation}
      Desarrollando, obtenemos:
      \begin{equation}
        S(n) = 4 \sum^{n+1}_{i=1} \frac{1}{i}
      \end{equation}
      
      Claramente, esto es $S(n)$ es $\mathcal{O}(log(n))$. Entonces: \\
      $T(n) = S(n) (n+1) = \mathcal{O}(log(n)) (n+1) = \mathcal{O}(nlog(n))$

\section {Compraventa de acciones}
\subsection{Algoritmo}
	% (Comentario Mario)
	% ¿Necesitamos suponerlos double? De hecho, con que sean datos donde haya definida relación de orden...
	Dado un array con valores numéricos, queremos calcular las posiciones a y b que nos den la mayor diferencia $array[b]-array[a]$, con la condición de que la posición b esté más a la derecha que a.
	Simularemos de esta forma una operación de compraventa de acciones en la que queremos obtener la mayor ganancia, donde el array representa el precio de las acciones a lo largo de un periodo de tiempo, y tenemos la natural restricción de que hemos de comprar las acciones antes de venderlas. 
	
	Vamos a emplear un algoritmo recursivo, utilizando la técnica divide y vencerás. 
	La idea es la siguiente: 
	
	Como caso base, si el array tiene longitud 1, devolvemos la posición del elemento. 
	
	Si el array tiene más de un elemento, lo dividimos en dos mitades. Se nos presentan tres disyuntivas:
	% Técnicamente es *una* disyuntiva con tres situaciones, no? -David
	\begin{itemize}
		\item Los valores minimal\footnote[1]{Llamamos minimal al valor al que compramos las acciones. No debe confundirse con el mínimo del array. De igual forma, llamamos maximal al valor al que vendemos las acciones, que no debe confundirse con el máximo.}
			 y maximal están en la primera mitad.
		\item Los valores minimal y maximal están en la segunda mitad. 
		\item El valor minimal está en la primera mitad, y el maximal en la segunda. 
	\end{itemize}
	
	Para resolver esto, llamaremos recursivamente a la función sobre ambas mitades, y calcularemos el mínimo en la primera mitad, y el máximo en la segunda. Compararemos entre los tres resultados para quedarnos con la mayor ganancia y finalmente devolver las posiciones correspondientes. 
	
	\begin{algorithm}[H]
		\begin{algorithmic}[1]
			\REQUIRE \ \\
	        	\texttt{Array}\\\
	     	\IF{\texttt{Array.size == 1}}
	     		\RETURN {Posición del elemento como momento de compra y venta}
	     	\ELSE
				\STATE{Dividimos el array en dos}
	     		\STATE{Definimos: \\
	     			\texttt{L}=\texttt{Mitad izquierda del array}\\
	     			\texttt{R}=\texttt{Mitad derecha del array}\\}
	     		\STATE{Llamamos recursivamente a la función sobre L y R}
	     		\STATE{Calculamos: \\ 
	     			\texttt{Min}=\texttt{Mínimo de L}\\
	     			\texttt{Max}=\texttt{Máximo de R}\\}
	     		\STATE{Hallamos la mayor de las tres ganancias y ajustamos los momentos de compra y venta}
	     		\RETURN{Momentos de compra y venta}
	    	\ENDIF
		\end{algorithmic}
		\caption{Compraventa de Acciones}
		\label{algoritmo}
	\end{algorithm}

\subsection{Análisis de eficiencia}
Nuestro algoritmo tiene dos llamadas recursivas con un argumento de tamaño $\frac{n}{2}$, siendo $n$ el tamaño actual. Además, en cada llamada recursiva hemos de calcular un máximo y un mínimo, operaciones que consumen un tiempo lineal. Nuestro caso base es de tiempo constante, lo que nos da la recurrencia: 
\begin{equation}
T(n)=\left\lbrace
	    \begin{array}{lr}
            1 & n=1\\
            2T\left(\frac{n}{2}\right) + n & n>1\\
            \end{array}
	    \right.
\end{equation}

Si cambiamos de variable, $m = log_2(n);\ n = 2^m$, y resolvemos la ecuación homogénea asociada:
$$ x_m-2x_{m-1}=0 \Rightarrow x_m = 2^m\ c_1 $$
Hallamos una solución particular para la ecuación en recurrencia no homogénea, como la sucesión $\{0, 2, 8, 24, 64\dots\}=\{m2^{m}\}_{m\ge 0}$ que verifica la ecuación en recurrencias. La solución general es por tanto:
$$ x_m = 2^m\ c_1 + m2^{m}\ \forall m\ge 0$$
Al deshacer el cambio de variable tenemos:
$$ T(n) = nc_1 + nlog(n)\ \in \mathcal{O}(nlog(n))$$
Podemos apuntar que, como esta recurrencia es la misma que la obtenida en el algoritmo de ordenación Mergesort, podríamos decir de antemano que la eficiencia de este algoritmo es $\mathcal{O}(n log(n))$.

\subsection{Algoritmo mejorado}

	El algoritmo anterior mejora a un algoritmo directo, que tendría una eficiencia cuadrática. Sin embargo, se puede ver que el cálculo del mínimo y el máximo de las dos partes del array en cada iteración nos impide obtener un algoritmo aun más eficiente, lo que nos lleva a preguntarnos si existe alguna forma de evitar esto. La respuesta es afirmativa; podemos aprovechar la técnica divide y vencerás para calcular también el máximo y el mínimo de un array, aprovechando las llamadas recursivas que hacemos. De esta forma: 
	
	\begin{itemize}
		\item Si el array tiene un elemento, éste es el máximo y el mínimo.
		\item El mínimo de un array con más de un elemento es el mínimo de los respectivos mínimos de las dos mitades (Análogo para el máximo).
	\end{itemize}
	
	Ahora, el proceso de calcular el máximo y el mínimo será constante en lugar de lineal, pues simplemente calculamos el mínimo o máximo de dos números obtenidos anteriormente gracias a la recursividad. 
	
	Nuestro algoritmo sería ahora de la siguiente forma: 
	
	\begin{algorithm}[H]
		\begin{algorithmic}[1]
			\REQUIRE \ \\
				\texttt{Array}\\\
			\IF{\texttt{Array.size == 1}}
				\RETURN {Posición del elemento como momento de compra y venta. Minimo = elemento. Máximo = elemento.}
			\ELSE
				\STATE{Dividimos el array en dos}
				\STATE{Definimos: \\
					\texttt{L}=\texttt{Mitad izquierda del array}\\
					\texttt{R}=\texttt{Mitad derecha del array}\\}
				\STATE{Llamamos recursivamente a la función sobre L y R. De esta forma también obtenemos el mínimo de L y el máximo de R}
				\STATE{Hallamos la mayor de las tres ganancias y ajustamos los momentos de compra y venta}
				\STATE{Recalculamos el mínimo y el máximo del array}
				\RETURN{Momentos de compra y venta. Mínimo y máximo.}
			\ENDIF
			\end{algorithmic}
		\caption{Compraventa de acciones mejorado}
		\label{algoritmo mejorado}
	\end{algorithm}
	
\subsection{Análisis de eficiencia}
El nuevo algoritmo sigue teniendo dos llamadas recursivas con un argumento de tamaño $\frac{n}{2}$, siendo $n$ el tamaño actual. Sin embargo, el nuevo calculo del mínimo y del máximo se hace entre dos números, lo que lo hace de tiempo constante. Nuestro caso base sigue siendo de tiempo constante, lo que nos da la nueva ecuación en recurrencias: 
\begin{equation}
T(n)=\left\lbrace
	    \begin{array}{lr}
            1 & n = 1\\
            2T\left(\frac{n}{2}\right) + 1 & n>1\\
            \end{array}
	    \right.
\end{equation}
Cambiamos de variable, $m = log_2(n);\ n = 2^m$ y resolvemos la ecuación homogénea asociada: 
$$ x_m-2x_{m-1}=0 \Rightarrow x_m = 2^m\ c_1 $$
Hallamos una solución particular para la ecuación en recurrencia no homogénea, como la sucesión $\{0, 1, 3, 7, 15\dots\}=\{2^{m}-1\}_{m\ge 0}$ que verifica la ecuación en recurrencias. La solución general es por tanto:
$$ x_m = 2^m\ c_1 + 2^{m}-1\ \forall m\ge 0$$
Al deshacer el cambio de variable tenemos:
$$ T(n) = nc_1 + n-1 = n(c_1 + 1)-1\ \in \mathcal{O}(n)$$
	
\end{document}